
       \let\stop\empty
       \documentclass[zihao = -4]{exam-zh}
       % !TeX root = ./exam-zh-chinese.tex

       \usepackage{graphicx}


       \examsetup{
         page = {
           size            = a3paper,
           show-columnline = false,
           foot-content    = {语文试题第;页(共8页)}
         },
         font = times,
         fillin/no-answer-type = none,
         sealline = {
           show        = true,
           scope       = mod-2,
           circle-show = false,
           line-type   = solid,
           odd-info-content = {
             {\heiti \zihao{4}姓名} {\underline{\hspace*{8em}}},
             {\heiti \zihao{4}准考证号} {\examsquare{9}},
             {\heiti \zihao{4}考场号} {\examsquare{2}},
             {\heiti \zihao{4}座位号} {\examsquare{2}},
           },
           odd-info-xshift = 12mm,
           text = {此卷只装订不密封},
           text-width  = 0.98\textheight,
           text-format = \zihao{-3}\sffamily,
           text-xshift = 20mm
         },
         square = {
           x-length = 1.8em,
           y-length = 1.6em
         },
         select = {
           pre-content  = {},
           post-content = {},
           separator    = \qquad
         }
       }


       \ctexset{
         subsection = {
           number     = \chinese{subsection},
           name       = {(,)},
           aftername  = {},
           format     = \bfseries,
           indent     = 2\ccwd,
           beforeskip = .25\baselineskip,
           afterskip  = .25\baselineskip
         },
         subsubsection = {
           numbering  = false,
           indent     = 2\ccwd,
           format     = \sffamily\bfseries,
           beforeskip = .25\baselineskip,
           afterskip  = .25\baselineskip
         }
       }


       \xeCJKsetup{
         underdot = {
           depth  = 0.4em,
           format = \large,
           boxdepth = 0pt
         },
         underline = {
           thickness = 0.8pt
         }
       }

       \setlist[enumerate, 2]{
         left       = 0pt,
         labelsep   = 0pt,
         label      = {(\arabic *)}
       }


       \title{2024年普通高等学校招生考试统一模拟演练}

       \subject[3.5em]{语文}

       \AtEndPreamble{
         \geometry{
           % showframe
           margin  = 0.9in,
           inner   = 1.25in,
           outer   = 1.2in,
         }
       }


       \begin{document}

       \secret

       \maketitle
       本试卷共 8 页,23 题。全卷满分 150 分。考试用时 150 分钟。


       \begin{notice}
         \item 答题前,先将自己的姓名、准考证号、考场号、座位号填写在试卷和答题卡上,
           并将准考证号条形码粘贴在答题卡上的指定位置。
         \item 选择题的作答:每小题选出答案后,用 2B 铅笔把答题卡上对应题目的答案标号涂黑。
           写在试卷、草稿纸和答题卡上的非答题区域均无效。
         \item 填空题和解答题的作答:用黑色签字笔直接答在答题卡上对应的答题区域内。
           写在试卷、草稿纸和答题卡上的非答题区域均无效。
         \item 考试结束后,请将本试卷和答题卡一并上交。
       \end{notice}



       \section{现代文阅读(35 分)}

       \subsection{现代文阅读 I(本题共 5 小题,19分)}

       阅读下面的文字,完成1~5题。

       \begin{material}
       在华南国家植物园的生物园里,洋金凤不惧酷暑花开得正艳。近年来作为成功培育的园艺植物,洋金凤已被培育出多个品种并广泛用于园林绿化。而今广州处处可见“蛱蝶花”的娇美身影,在“亲戚”凤凰木花期结束后,接力延续夏日之美。  专家提醒,虽然洋金凤开花婀娜动人,美丽表象下却是花谢后的“毒”,种子虽有药用价值但不建议轻易采摘。  两招教你分清“金凤”和“洋金凤”  初夏时节的广州,凤凰木的花朵曾以其绚烂之火红渲染一座城,随着盛夏气温的升高,人们仿佛又见凤凰木花朵的摇曳身姿,悄然绽放在城市的每一个转角之处。纵使凤凰木花期已过,但与其品相极度相似的“洋金凤”早已粉墨登场,将这份火热之情延续到夏末。  不少人分不清楚凤凰木和洋金凤,因为无论是叶子排列的序列和花型花色都极为相似。尤其凤凰木和洋金凤在广东地区都属于非常常见的树木,所以很多人都会将二者混淆,甚至在很多地方,凤凰木的俗名就叫作“金凤花”。那么“金凤”和“洋金凤”究竟如何分辨呢?华南国家植物园李世晋研究员告诉记者,这两者确实有着许多共同之处,它们都属于豆科,但是分属于不同的属。洋金凤属于“小凤花属”,来自南美洲;凤凰木则属于“凤凰木属”,来自非洲。  对于不少广州老街坊而言,高大挺拔的凤凰木是学生时代的回忆,因此“身高”就是将其与洋金凤区分的最好特征。据李世晋介绍,洋金凤为灌木或小乔木,呈绿色或粉绿色,茎干上有一些稀疏的尖刺;而凤凰木则可以长成大乔木。细心观察,洋金凤虽然个子不高,树干不似凤凰木那般“孔武有力”,但胜在树形轻盈婀娜,而且和凤凰木一样拥有羽状复叶。由于洋金凤和凤凰花都属豆科植物,所以不仅花朵长得像,连叶子都很像。洋金凤的叶子是二回羽状复叶,在纤细的枝条上,7-12对长圆形或倒卵形的小叶对生,每每随风起舞姿态都优美动人。  区分凤凰木和洋金凤的技巧,除了“身高”还有“颜值”。两者的花朵就像孪生姐妹,洋金凤简直就是灌木版的凤凰花,然而仔细观察会发现,洋金凤的花形虽然没有凤凰花大,但花蕊非常细长,总10枚长短不一的雄蕊和1枚雌蕊,以傲然之姿远远伸出花冠外。黄色或橙红色的花瓣边缘呈波状皱褶,开花时花瓣犹如色彩艳丽的小蒲扇;未开花时,花蕾看上去则像一个个长形的小汤勺,“盛放”着未开放的花瓣和花蕊。  而这份色彩奔放的热情,正是来自阳光灿烂的南美洲 ,“巴巴多斯”岛。  从“云实属”到“小凤花属”的归属历程  洋金凤属于科学界很早就认知的物种,早在中国科学院华南植物园陈焕镛院士参与编写的《广州植物志》中就有记载。据了解,在过去的《中国植物志》上,洋金凤的中文名也叫金凤花,然而除了凤凰木别名为“金凤花”,在“凤仙花科”还有一种花也叫金凤花,多次出现了重名的情况。所以在最新的植物志上,它被统一成“洋金凤”。既生瑜何生亮,站在植物分类学的角度,洋金凤的故事也很有趣。  据李世晋介绍,1753年,瑞典博物学家林奈最早将洋金凤取名——拉丁学名“Poinciana pulcherrima L.”,其种加词取自拉丁文“pulcher”,意为“美丽的”。到了1791年,洋金凤又被另一位瑞典植物学家转移到“Caesalpinia属”,即普遍认可的“云实属”,也可以称为“广义云实属”。然而很快人们就发现,广义云实属内部关系过于复杂,为了将它们的关系理顺,广义云实属需要被拆分成几个小属。“就这样,经过多年来植物学家们的研究和探讨,最终将洋金凤所属的中文名拟为‘小凤花属’,所以我们今天可以说金凤花为‘小凤花属’的一个成员。”  可以说,最终归属于“小凤花属”的洋金凤以花取胜,尤其当花朵盛开时花瓣五裂的姿态,鲜橙红色的花瓣犹如蝶翼一般,长长的花丝就像蝴蝶的触角,洋金凤又被称为“红蝴蝶”。这并不是它唯一的“昵称”,除了蝴蝶花、黄蝴蝶、番蝴蝶之外,还有孔雀花、蛱蝶花、红天堂鸟、墨西哥天堂鸟……其中最特别的名字非“巴巴多斯的骄傲”莫属。  据了解,洋金凤是加勒比岛国巴巴多斯的国花,曾有研究显示其原产于西印度群岛,但由于普遍栽培,其确切的起源地未知。“洋金凤是全世界热带地区广泛种植的观赏植物,在我国南方各地庭园也常被栽培,已经是很成功的园艺植物了。”李世晋告诉记者,目前华南国家植物园对豆科植物的研究主要集中在分类学、固氮生理学等方向的研究。  而今在广州,除了在华南国家植物园的生物园和木本花卉区可以欣赏到洋金凤的婀娜多姿,也能通过一片片碧绿的羽叶,在不少道路的两侧寻觅到“蝴蝶”的身影。  虽有药用价值,不建议轻易采摘  叶如飞凰之羽,花若丹凤之冠,不以色香引人,却以姿容形态取胜。纵使颜值很高,洋金凤起初存在的“价值”却为药用,最初在药用植物园种植培育。据了解,洋金凤的种子可榨油及药用,根、茎、果均可入药。在《中国中药资源志要》曾记载关于洋金凤的药用价值,根有强壮、通经、收敛之效,树皮有收敛、止泻、通经之效,叶子可以解热、通经、泻下、驱虫,花则可以用来解热、活血、止咳。  然而洋金凤属于“毒性中药材”,作为轻泻剂使用的叶子能引致小产,而根也含有毒性。据《香港有毒植物图鉴》显示,曾有报道进食云实属植物的种子后中毒的案例,患者多为儿童,常见出现恶心、呕吐、腹泻、腹痛的症状,严重中毒可导致脱水。“据我所知,许多豆科植物的果实和种子具有一定的毒性。如我们日常食用的豇豆和四季豆也具有一定毒性,在烹饪时一定要充分煮熟。”李世晋说道。  洋金凤属于热带植物,酷热的天气、充沛的雨水都是它开花繁衍的条件之一,从8月开始,这些蝴蝶般的花朵就会越开越旺。行走城市街道时,在成熟开裂掉落地上的荚果里,能看到种子一枚枚整齐地排列在内壁,看起来有点像瓜子仁。李世晋提醒,关于洋金凤的种子究竟有多大的毒性,未有正式研究结果之前,市民们切莫轻易尝试。
\\
       \end{material}


       \begin{question}
         下列对材料相关内容的理解和分析,不正确的一项是(3分)
         \begin{choices}
           \item  对比论证:通过比较藜麦价格上涨前后消费者的变化,展示了藜麦对当地居民的影响。 

           \item  举例论证:通过引用《独立报》和《纽约时报》的研究报告和报道,证明了藜麦价格上涨对玻利维亚和秘鲁人的影响。 

           \item  引用论证:通过引用经济学家马克·贝勒马尔的观点,对藜麦价格上涨的影响提出了不同的看法。 

           \item  数据论证:通过分析秘鲁家庭支出的调查数据,展示了藜麦贸易对当地居民生活水平的影响

         \end{choices}
       \end{question}

       \begin{question}
         下列对材料相关内容的理解和分析,不正确的一项是(3分)
         \begin{choices}
           \item 藜麦是一种营养丰富的食物,受到全球消费者的喜爱。

           \item 藜麦价格上涨使其成为一种奢侈品,让一些贫困人群难以承受。

           \item 由于藜麦价格上涨,玻利维亚和秘鲁的藜麦消费量下降,当地居民生活水平下降。

           \item 藜麦在全球市场上的需求不断增长,导致其价格不断攀升

         \end{choices}
       \end{question}

       \begin{question}
         根据材料内容,下列说法正确的一项是(3分)
         \begin{choices}
           \item  营养学家张教授说:“藜麦确实营养均衡,含有大量人体必需的氨基酸和矿物质,对素食者和肠胃疾病患者来说是个好选择。”

           \item  美国宇航局发言人表示:“藜麦是地球上营养最均衡的食物之一,适合宇航员食用,说明它的营养价值非常高。”

           \item   玻利维亚农民何塞说:“藜麦的价格上涨让我们富裕起来,我们用这些钱改善了生活,孩子们也能吃到更好的食物。”

           \item  秘鲁农民妇女玛利亚说:“藜麦是我们的传统作物,现在它为我们带来了更多的收入,让我们能够投资教育和健康

         \end{choices}
       \end{question}

       \begin{question}
         请简要说明文本中关于“藜麦价格上涨和藜麦消费量下降”的问题,文中哪些信息表明这一现象并非如西方媒体所报道的那样简单。
(4分)
       \end{question}

       \begin{question}
         作者如何运用不同的论证方法来反驳关于藜麦价格上涨对当地居民造成负面影响的观点?
(6分)
       \end{question}


       \subsection{现代文阅读 II(本题共4小题,16分)}

       阅读下面的文字,完成6~9题。


       \begin{material}[
         title = 比邻而居
, author = 耿立
,
       ]

       ①与炊烟和蚯蚓是邻居,木镇的人就像夫子而言:与德比邻,道不孤。木镇人有木镇人的道,远亲不如近邻,近邻不如对门,对门的是树和泥之河,是泥土上的一切。人们说乡村是泥土做的。是啊,木镇的一切都在泥土上。木镇有的人不识字,但不妨碍他们把泥土当作《圣经》,他们知道大地上 的一切都是泥土给的,炊烟呼吸,鸡叫驴滚,草木种子,都是圣经不同的文字。②如果说草的种子是汉语印制的,父亲能读懂;那村长折腾土地的脾气就是英文印制的,他读不懂。有时村长让大家种水稻,却颗粒无收,父亲说我们这里的地寒,水稻是金贵喜暖的玩艺,泥土有脾气,你不要拗,种子也有脾气,你能把庄稼种到石板上? ③父亲说泥土就如一令席子,植物动物 泥之河与人都在这令席子上。席子在家里要金贵地用,对土地,对土地上的一切,亦应如是,大家都是对门合户的,抬 头不见低头见,以免作错了事说错了话,脸红。④我看到父亲在田埂上掮着锄头走,一遇到牛从对面思索着过来,父亲就退后一步,不像西方的人把手捂着胸脯心跳的地方那样,但绝对的虔敬,如除夕从祖坟把先辈的神灵请回过年一样,父亲相信牛与人一样,离头三尺的地方有神灵。⑤我读过父亲的手,虽然如树皮一样皱褶苍老,有点变形,手上的青筋如蚯蚓,但他在泥土里多年操持,有着泥土的温暖,我一握的时候,就像庄稼的汁液传到我的脉管和血管,这是泥土的温度。父亲的手粗糙么?但这样的手在泥土里绝对灵活,他锄地时,绝对不伤害庄稼,而对草,也是尽量照顾,只要和庄稼和谐相处,父亲是不会对草痛下杀手的。    ⑥但父亲年老了,手指有时不太灵便,在一次给麦田松土的时候,不小心把一 根在泥土下路过的蚯蚓斩断了,父亲内疚喃喃:这怎么好,这怎么好。他停下手,拿眼睛乜斜地看我一下,从兜里掏出一只用烟叶卷成的烟,咝咝地点着,然后闭上眼睛,他说出了令我吃惊的话:让我装死一会。这是在推己及物想像蚯蚓的痛吗?⑦即使冬令时节,父亲也闲不住。父亲会把土墙上的野蜂窝盖上麦秸,怕小生灵跋涉不过雪季;他也常和叫做家贼的麻雀对话,有时就撒出一些苞谷犒赏一下这些小家伙,作为一年在窗前恪尽职守叫醒农人的报答;有时父亲要在阳光晴好的时候堆粪翻粪晒粪,这不是轻松活,这是为了对泥土来年的报偿,泥土在收获后,如产后的女人,你想他们陪伴着小麦走了一春,陪伴着苞谷走了夏季秋季,如今到了该歇息的时候,就如女人产后要吃红皮鸡蛋喝红糖水,父亲在把庄稼地腾出茬以后,就想着为泥土养身子了。到了秋收罢了,父亲还会到田地里去,他像逡巡的士兵,把泥土里的瓦块、砖头剔除掉,怕这些骨头硌着睡眠的泥土,怕在地里漫游的小动物们闪了腰,怕来年开春撞坏了犁耙。父亲心里最清楚,土地糊弄不得,土地和人是兄弟,多少辈子都比邻而居,对土地好也是对自己好。⑧从地里回来的父亲脸上有一块泥巴,母亲想用手抠下,接着就想卷起衣襟擦,父亲招呼了一下说不用了,是当着我的面,父亲羞涩了,但母亲的亲昵是对劳作的一种尊重,泥土在脸上怎么了,有时米粒和碎馍掉到地上,虽满是泥,父亲吹一下,或者母亲用衣襟擦一下,就填到嘴里。泥土在父亲的脸上,是土地的徽章么,作为对一辈子的老邻居的奖赏,是否在父亲的脸上撒一把草籽,用洗脸水一浇就能发芽?诗人雅姆说:如果脸上有泥的人从对面走来,要脱帽致敬,先让他们过去。⑨是啊,我们什么时候,对有泥的人有过足够的尊重呢?我们向泥土敬个礼吧。(选自2012年3月30日《菏泽青年作家博客》,有删改)11.简析第一段划线句在文中的作用。(3分)12.父亲“金贵”“土地”和“土地上的一切”主要表现在哪些方面?请作简析。(3分)13.赏析第7段划曲线的句子。(5分)14.文末引用雅姆的诗句有何作用?(3分)本试题由一苇轩(高中语文题库)www.gzywtk.com进行考点归类细化整理15.联系全文,解说“我们向泥土敬个礼”的深刻内涵。(6分)
\\
       \end{material}


       \begin{question}
         下列对小说相关内容的理解,正确的一项是(3 分)
         \begin{choices}
           \item  父亲在信中告诉儿子,他的成长过程中会经历许多不同的阶段,其中之一就是进入大学,并鼓励儿子在第一个暑假去旅行,体验生活。

           \item  父亲详细描述了儿子在旅行途中可能会经过的地方,如芜湖、江堤、村子、河边等,这些地方都富有生活气息,让人感受到乡村的宁静和友好。

           \item  父亲提醒儿子在村子中要找到与他父亲当年相似的人,他们可能会记得父亲当年的点点滴滴,并告诉儿子一些有关父亲的故事。

           \item  父亲建议儿子在旅行时带走一条狗,这表明狗在父亲的心中有着特殊的地位,也许是他当年乡村生活中的一部分

         \end{choices}
       \end{question}

       \begin{question}
         下列对小说艺术特色的分析鉴赏,不正确的一项是(3 分)
         \begin{choices}
           \item  社戏是在秋季举行的,与文章中提到的“九月二十二”和“适逢秋天橘子收获时节”相呼应。 

           \item  社戏是萝卜溪村一年一度的重大活动,受到村民的重视和期待。 

           \item  社戏的演出地点在伏波宫前空坪中,演出时间为六天,符合往年成例。 

           \item  社戏期间,村里人会穿上新衣服,携带零用钱,类似于过盛大节日
  
         \end{choices}
       \end{question}

       \begin{question}
         “我看到了你的黑黑的人影,我的心里充满了慌乱。”这句话描绘了多重的情感体验。请加以梳理概括。
(4分)
       \end{question}

       \begin{question}
         读书小组要为此文写一则文学短评。经讨论,甲组提出一组关键词:未来·回忆·成长;乙组提出一个关键词:河流。请任选一个小组加入,围绕关键词写出你的短评思路。
(6分)
       \end{question}



       \section{古代诗文阅读(35 分)}

       \subsection{文言文阅读(本题共 5 小题,20 分)}

       阅读下面的文言文,完成10~14题。

       \begin{material}[source = (节选自《七十列传·孙子吴起列传原文
》)]

         孙子武者,齐人也。以兵法见於吴王阖庐。阖庐曰:“子之十三篇,吾尽观之矣,可以小试勒兵乎?”对曰:“可。”阖庐曰:“可试以妇人乎?”曰:“可。”於是许之,出宫中美女,得百八十人。孙子分为二队,以王之宠姬二人各为队长,皆令持戟。令之曰:“汝知而心与左右手背乎?”妇人曰:“知之。”孙子曰:“前,则视心;左,视左手;右,视右手;後,即视背。”妇人曰:“诺。”约束既布,乃设鈇钺,即三令五申之。於是鼓之右,妇人大笑。孙子曰:“约束不明,申令不熟,将之罪也。”复三令五申而鼓之左,妇人复大笑。孙子曰:“约束不明,申令不熟,将之罪也;既已明而不如法者,吏士之罪也。”乃欲斩左右队长。吴王从台上观,见且斩爱姬,大骇。趣使使下令曰:“寡人已知将军能用兵矣。寡人非此二姬,食不甘味,原勿斩也。”孙子曰:“臣既已受命为将,将在军,君命有所不受。”遂斩队长二人以徇。用其次为队长,於是复鼓之。妇人左右前後跪起皆中规矩绳墨,无敢出声。於是孙子使使报王曰:“兵既整齐,王可试下观之,唯王所欲用之,虽赴水火犹可也。”吴王曰:“将军罢休就舍,寡人不原下观。”孙子曰:“王徒好其言,不能用其实。”於是阖庐知孙子能用兵,卒以为将。西破强楚,入郢,北威齐晋,显名诸侯,孙子与有力焉。孙武既死,後百馀岁有孙膑。膑生阿鄄之间,膑亦孙武之後世子孙也。孙膑尝与庞涓俱学兵法。庞涓既事魏,得为惠王将军,而自以为能不及孙膑,乃阴使召孙膑。膑至,庞涓恐其贤於己,疾之,则以法刑断其两足而黥之,欲隐勿见。
\\

       \end{material}


       \begin{question}
         以下句子有三处需要断句,请用铅笔将答题卡上相应位置的答案标号涂黑\\
        韩非书云夫子善之引以张本然后难之岂有不似哉?

       \end{question}

       \begin{question}
         下列对文中加点词语的相关内容的解说,不正确的一项是(3 分)
         \begin{choices}
           \item 武臣,指文章中的陈人武臣,他提出了对韩非子观点的看法。 

           \item 子鲋,指武臣的儿子,他回应武臣的观点并提出了自己的见解。 

           \item 具臣,指赵襄子行赏时先加封的臣子,子鲋用此例来说明韩非子书中的观点。 

           \item 夫子,在文中指孔子,子鲋用孔子的卒年来反驳韩非子对历史的记载
  
         \end{choices}
       \end{question}

       \begin{question}
         下列对原文有关内容的概述,不正确的一项是(3 分)
         \begin{choices}
           \item 武臣认为韩非的立法与孔子的论点有许多不同之处,他认为在遏制邪恶和鼓励善行方面,韩非的观点可能与孔子相当。

           \item 子鲋指出,当今世人喜欢用极端的比喻来描述高低,并且喜欢引用经典和圣人的话来提升自己的形象,以赢得他人的信任。

           \item 子鲋举例说明,赵襄子在行赏时,先赏赐那些有具体贡献的臣子,后赏赐那些有功之人,这与韩非书中的描述相矛盾,显示出韩非的记载可能有诈。

           \item  子鲋批评武臣相信诸子百家的学说,认为这是错误的,因为诸子书中的内容都是如此,他举例说明韩非的记载与历史事实不符,以此质疑韩非的可靠性

         \end{choices}
       \end{question}

       \begin{question}
         把文中画横线的句子翻译成现代汉语。(8 分)
         \begin{enumerate}
           \item 今世人有言高者必以极天为称,言下者必以深渊为名。

           \item 昔我先君以春秋哀公十六年四月己丑卒,至二十七年荀瑶与韩、赵、魏伐郑,遇陈恒而还,是时夫子卒已十一年矣,而晋四卿皆在也

         \end{enumerate}
       \end{question}

       \begin{question}
         子鲋如何反驳武臣将韩非视为“世之圣人”的观点?
(3分)
       \end{question}


       \subsection{古代诗歌阅读(本题共 2 小题,9 分)}

       阅读下面这首唐诗,完成 15~16 题。


       \begin{poem}[author = 杜牧
, title = 赠别二首·其一
]
         娉娉袅袅十三余,豆蔻梢头二月初。\\
春风十里扬州路,卷上珠帘总不如。\\
       \end{poem}

       \begin{question}
         下列对这首诗的理解和赏析,不正确的一项是(3 分)
         \begin{choices}
           \item  “万壑树参天,千山响杜鹃”:形象地描绘了梓州山林的壮丽景象,万壑千山,树木参天,杜鹃声声,生动再现了自然的宏伟与生机。 

           \item  “山中一夜雨,树杪百重泉”:通过“一夜雨”和“百重泉”的描写,表现了山中气候的多变与山泉的壮丽,雨后山泉从树梢倾泻而下,形成百泉争流的景象。 

           \item   “汉女输棂布,巴人讼芋田”:反映了当时蜀地百姓的生活情景,汉女织布纳税,巴人争论田地,体现了诗人对百姓生活的高度关注。 

           \item  “文翁翻教授,不敢倚先贤”:诗人以文翁喻李使君,期望李使君能够发扬前人的教育事业,翻新教化,而不依赖于前人的成就,显示了诗人对李使君的期望与鼓励

         \end{choices}
       \end{question}

       \begin{question}
         请赏析诗句“山中一夜雨,树杪百重泉”中的情感。(6分)
       \end{question}


       \subsection{名篇名句默写(本题共 1 小题,6 分)}

       \begin{question}
         补写出下列句子中的空缺部分。(6分)
         \begin{enumerate}
           \item 1. 在《离骚》中,“\fillin[width = 5em][]”两句通过描绘自然景象,表达了诗人追求真理的坚定决心和不懈努力。

           \item 2. 在《登高》中,“\fillin[width = 5em][]”两句通过描绘山水的壮丽,反映了诗人对国家命运的关切和对人民疾苦的同情。

           \item 3. 在《赋得古原草送别》中,“\fillin[width = 5em][]”两句以简洁的语言描绘了离别时的情感,表达了诗人对友人的深深思念和不舍之情。

         \end{enumerate}
       \end{question}



       \section{语言文字运用(20 分)}

       \subsection{语言文字运用 I(本题共 3 小题,11 分)}

       阅读下面的文字,完成18~20题。

       \begin{material}
        遥想当年,\underline{\hspace*{3.5em}},我们曾满怀激情,勇往直前。岁月如梦,匆匆而过,那些曾经的日子,犹如\underline{\hspace*{3.5em}}。如今,我们感叹时光荏苒,青春不再,但心中那份执着仍然熠熠生辉。在这个充满挑战和机遇的时代,我们更应该奋发向前,勇攀高峰。无论前路如何坎坷,我们都不能放弃信念和希望。正如古语所说:“\underline{\hspace*{3.5em}},不坠青云之志。”我们要时刻保持清醒的头脑,勇敢地面对生活的风风雨雨。有时候,我们也会感到迷茫和彷徨,仿佛迷失在人生的十字路口。但请不要忘记,阳光总在风雨后,我们要相信自己,坚定地走自己的路。正如成语所说:“殊途同归”,只要我们心中有梦想,终会找到属于自己的方向。\\
       \end{material}

       \begin{question}
       请在文中横线处填入适当的成语
       \end{question}

       \begin{question}
         请修改以下病句:我们感叹时光荏苒,青春不再,但心中那份执着仍然熠熠生辉。

       \end{question}

       \begin{question}
         请赏析下面的句子:有时候,我们也会感到迷茫和彷徨,仿佛迷失在人生的十字路口。

       \end{question}


       \subsection{语言文字运用 II(本题共 2 小题,9 分)}

       \examsetup{
         fillin = {
           no-answer-type = counter,
           no-answer-counter-label = \circlednumber*
         }
       }

       阅读下面的文字,完成21~22题。

       \begin{material}
         青春,如同清晨的阳光,温暖而耀眼;青春,似那奔腾的江河,激情澎湃。站在青春的门槛上,我们怀揣着梦想,勇往直前。青春的天空,淡淡的蓝,镶嵌着洁白的云朵。我们在这样的天空下,笑声洒满每个角落。青春的我们应该拥有坚定的信念,不畏困难,砥砺前行。那些无数个日夜里,我们一起奋斗过的痕迹,成为了我们最宝贵的财富。青春的我们,拥有无尽的活力,仿佛是一个个跳跃的音符,奏响生命中最美的旋律。我们在阳光下奔跑,感受风的拥抱,倾听心的声音。这美好的青春时光,让我们用心去感受,去珍惜。\\
       \end{material}

       \begin{question}
         请根据上下文补全句子:青春,如同清晨的阳光,温暖而耀眼;青春,\fillin,激情澎湃。青春的天空,淡淡的蓝,镶嵌着洁白的云朵。我们在这样的天空下,\fillin。青春的我们,拥有无尽的活力,仿佛是一个个跳跃的音符,\fillin。

       \end{question}


       \begin{question}
         请解释下面词语的表层含义和深层含义:(1)早晨的阳光:,。(2)砥砺前行:,。

       \end{question}



       \section{写作(60 分)}

       \begin{question}
         阅读下面的材料,根据要求写作。(60分)
       \end{question}

       \begin{material}
         "教育不是灌输,而是点燃火焰。" 
---- 威廉·巴特勒·叶芝
\\
         "真正的教育是激发人的潜能,培养人的兴趣。" 
---- 爱因斯坦
\\
       \end{material}

       以上论述具有启示意义。请结合材料写一篇文章,体现你的感悟与思考。

       要求:选准角度,确定立意,明确文体,自拟标题;不要套作,不得抄袭;不得泄露个人信息:不少于 800 字。

       \end{document}
       